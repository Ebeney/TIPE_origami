\documentclass[a4paper,12pt,french]{report}  
\usepackage{babel} 
\usepackage[T1]{fontenc} 
\usepackage[utf8]{inputenc} 
\usepackage{lmodern}
%\usepackage{amsthm}
\usepackage{stmaryrd}
\usepackage{amsmath}
\usepackage{amssymb}
\usepackage{mathrsfs}
\usepackage{geometry}
\usepackage{graphicx}
\geometry{hmargin=2.5cm,vmargin=2cm}

\newtheorem{ax}{Axiome}[section]
\newtheorem{lemma}{Lemme}[section]
\newtheorem{proposition}{Proposition}[section]
\newtheorem{theorem}{Proposition}[section]
\newtheorem{proof}{Démonstration}[section]
\newtheorem{definition}{Définition}[section]
\newtheorem{exemple}{Éxemple}[section]


\title{TIPE : Origami et constructibilité}

\begin{document}
\maketitle
\renewcommand{\contentsname}{Sommaire}
\tableofcontents{}

\part{Éléments d'introduction}
\chapter{Introduction à la constructibilité : règle et compas}
	
	\section{Définitions}
		Il est nécessaire de commencer notre étude en posant quelques définitions et résultats élémentaires qui nous seront utiles par la suite.
		\subsection{Outils d'algèbre}
			
			\begin{definition}[Corps] 
				Un corps est une structure algébrique \( (\mathbb{K} , + , \times ) \) telle que :
				\begin{enumerate}
					\item Les lois \( + \) et \( \times \) sont des applications \( \mathbb{K}^2 \longrightarrow \mathbb{K} \)
					\item \((\mathbb{K} , +)\) soit un groupe commutatif, c'est-à-dire :
						\[ 
							\left\{ 
							\begin{array}{lll}
								\exists \, 0_{\mathbb{K}} \in \mathbb{K} , \forall x \in \mathbb{K}, x + 0_{\mathbb{K}} = 0_{\mathbb{K}} + x = x
								\\
								\forall x \in \mathbb{K} , \exists k \in \mathbb{K} , x + k = k + x = 0_{\mathbb{K}} \mbox{  et on notera } -x = k
								\\
								\forall (x,y) \in \mathbb{K}^{2},
											x + y = y + x
										
								\\
								\forall (x, y, z) \in \mathbb{K}^3, (x + y) + z = x + (y + z)
									 
							\end{array}
							\right.
						\]
					\item \((\mathbb{K}\setminus\{0\}, \times) \) est un groupe commutatif :
						\[ 
								\left\{ 
								\begin{array}{lll}
									\exists \, 1_{\mathbb{K}} \in \mathbb{K} , \forall x \in \mathbb{K}, x \times 1_{\mathbb{K}} = 1_{\mathbb{K}} \times x = x
									\\
									\forall x \in \mathbb{K} , \exists k \in \mathbb{K} , x \times k = k \times x = 1_{\mathbb{K}} \mbox{  et on notera } x^{-1} = k
									\\
									\forall (x,y) \in \mathbb{K}^{2},
												x \times y = y \times x
									\\
									\forall (x, y, z) \in \mathbb{K}^3, (x \times y) \times z = x \times (y + z)
										 
								\end{array}
								\right.
							\]
					\item La loi \( + \) est distributive sur la loi \( \times \) :
						\[
						\forall (x, y, z) \in \mathbb{K}^3, x \times (y + z) = (y + z) \times x = xy + xz
						\]
				\end{enumerate}
			
			\end{definition}
			
			
			\begin{definition}[Espace Vectoriel]
				Un \(\mathbb{K}\)-espace vectoriel \(E\)  est une structure algébrique \( (E, + , \cdot)\) telle que
				\begin{enumerate}
					\item \( (E, +) \) est un groupe commutatif
					\item La loi externe \( \cdot \) est une application \( \mathbb{K} \times E \longrightarrow E \) qui vérifie les axiomes suivants :
						\begin{enumerate}
							\item Pseudo-associativité :
								\(
								\forall (\lambda, \mu, x) \in \mathbb{K}^2 \times E, \mu(\lambda x) = (\mu \lambda) x
								\)
							\item Pseudo-distributivité :
								\(
								\left\{
									\begin{array}{ll}
										\forall (\lambda, \mu, x) \in \mathbb{K}^2 \times E (\lambda + \mu)x = \lambda x + \mu x \\
										\forall (\lambda, x, y) \in \mathbb{K} \times E^2 \lambda(x + y) = \lambda x + \lambda y
									\end{array}
								\right.
								\)
							\item Opérateur neutre :
								\(
								\forall x \in E, 1_\mathbb{K} \cdot x = x
								\)
						\end{enumerate}
				\end{enumerate}
				
			\end{definition}
			
			
			\begin{definition}[Extension de corps]
			
				Si \( \mathbb{K} \) et \( \mathbb{L} \) sont deux corps tels que \(\mathbb{K} \subset  \mathbb{L} \), alors
				\(\mathbb{L}\) est un \(\mathbb{K}\)-espace vectoriel. Dès lors, \(\mathbb{L}\) est appelée une \emph{extension} de \(\mathbb{K}\).
			
				\begin{proof}
					La structure d'espace vectoriel de \(\mathbb{L}\) s'hérite de sa structure de corps.
					\begin{enumerate}
						\item \(\mathbb{L}\) est un corps donc, en particulier, \((\mathbb{L}, +)\) est un groupe commutatif.
						\item Tout élément de \(\mathbb{K}\) étant en particulier un élément de \(\mathbb{L}\), il suffit de définir la loi
						externe \(\cdot\) à l'aide de la loi \( \times \) de \(\mathbb{L}\) pour que tous les axiomes de la loi externe d'un espace vectoriel s'héritent directement des propriétés de la loi \( \times \) d'un corps:
							\[
							\forall (x, y) \in \mathbb{K} \times \mathbb{L}, x \cdot y = x \times y
							\]
					\end{enumerate}
				\end{proof}
			\end{definition}
			
			
			
			
			\begin{definition}[Degré d'une extension]
				Soit \(\mathbb{L}\) une extension d'un corps \(\mathbb{K}\). Si la dimension de \(\mathbb{L}\) en tant que \(\mathbb{K}\)-espace vectoriel est finie, on l'appelle le \emph{degré} de l'extension, notée :
				\[
				[\mathbb{L}:\mathbb{K}] = \dim_\mathbb{K}\mathbb{L}
				\]
				S'il est égal à 2, on parlera d'\emph{extension quadratique}.
			\end{definition}
			
			\begin{proposition}[Relation de Chasles sur le degré]
				Soit \(K\), \(L\), \(M\) trois corps tels que \( K \subset L \subset M\), et les extensions sont de degré fini. On a alors :
				\[
				[M:K] = [M:L]{\times} [L:K]
				\]
				\begin{proof}
					On pose tout d'abord :
					\[
					\left\{
						\begin{array}{ll}
							p = [M:L] \\
							n = [L:K]
						\end{array}
					\right.
					\]
					
					
					Ensuite, exhibons une base de \(M\) en tant que \(K\)-espace vectoriel afin d'obtenir sa dimension.
					Soit \((l_i)_{1 \leq i \leq n} \in L^n\) une base de \(L\) en tant que \(K\)-ev.
					Soit \((m_j)_{1 \leq j \leq p} \in L^n\) une base de \(M\) en tant que \(L\)-ev.
					
					Montrons que \((l_i m_j)_{(i, j) \in \llbracket 1, n \rrbracket \times \llbracket 1, p \rrbracket}\) est une base de M en tant que \(K\)-espace vectoriel.
					\begin{enumerate}
					\item \emph{Caractère générateur} : soit \(x \in M\), alors 
						\[
						\exists (\lambda_1, \dots, \lambda_p) \in L^p, \sum_{j  = 1}^{p} \lambda_j m_j = x
						\]
						Or \(\forall i \in \llbracket 1, p \rrbracket, \lambda_i \in L\), donc :
						\[ 
						\exists (\mu_{i, 1}, \dots, \mu_{i, n}) \in K^n, \sum_{j  = 1}^{n} \mu_{i, j} l_j = \lambda_i 
						\]
						Puis :
						\begin{align*}
						x 	&= \sum_{i  = 1}^{p} \lambda_i m_i = x \\
							&= \sum_{i  = 1}^{p} \left(\sum_{j  = 1}^{n} \mu_{i, j} l_j\right) m_i \\
							&= \sum_{(i, j) \in \llbracket 1, p \rrbracket \times \llbracket 1, n \rrbracket}\mu_{i, j} (l_j m_i)
						\end{align*}
					\item \emph{Caractère libre} : soit \((\lambda_{i, j})_{ (i, j) \in \llbracket 1, n \rrbracket \times \llbracket 1, p \rrbracket} \in K^{np}\) telle que :
							\[
							\sum_{(i, j) \in \llbracket 1, n \rrbracket \times \llbracket 1, p \rrbracket} \lambda_{i, j} (l_j m_i) = O
							\]
						Soit encore :
							\[
							\sum_{i=1}^{p} \underbrace{\left( \sum_{j = 1}^{n} \lambda_{i, j} l_j \right)}_{\in L}m_i = 0
							\]
						Donc, par liberté de \((m_1, \dots, m_p)\), on a :
							\[
							\forall i \in \llbracket 1, p \rrbracket, \sum_{j = 1}^{n} \underbrace{\lambda_{i, j}}_{\in K} l_j = 0
							\]
						Donc, par liberté de \((l_1, \dots, l_n)\), on obtient:
							\[
							\forall i \in \llbracket 1, p \rrbracket, \forall j \in \llbracket 1, n \rrbracket, \lambda_{i, j} = 0
							\]
					\item On a ainsi montré que \((l_i m_j)_{(i, j) \in \llbracket 1, n \rrbracket \times \llbracket 1, p \rrbracket}\) est une base du \(K\)-espace vectoriel \(M\). Or, cette base contient \(n\times p\) vecteurs, donc l'extension M est bien de degré \(n\times p\) par rapport à \(K\), ce qui achève la preuve. \(\square\)
					
					\end{enumerate}
				\end{proof}
			\end{proposition}
			
		\subsection{Nombres algébriques}
			\begin{definition}[Nombre algébrique]
			\end{definition}
			\begin{definition}[Degré et polynôme minimal]
			\end{definition}
			\begin{proposition}
			\end{proposition}
			\begin{proposition}
			\end{proposition}
		
	\section{Ensemble des nombres constructibles}
		\subsection{Théorème de Wantzel}
			\begin{theorem}
				
			\end{theorem}
			\begin{proof}
				
			\end{proof}
		
		\subsection{Exemples célèbres}




\end{document}
