\documentclass[a4paper,12pt,french]{report}  
\usepackage{babel} 
\usepackage[T1]{fontenc} 
\usepackage[utf8]{inputenc} 
\usepackage{amsmath}
\usepackage{geometry}
\usepackage{graphicx}
\geometry{hmargin=2.5cm,vmargin=2cm}

\newtheorem{ax}{Axiome}[part]
\newtheorem{lemma}{Lemme}[part]
\newtheorem{proposition}{Proposition}[part]
\newtheorem{theorem}{Proposition}[part]
\newtheorem{proof}{Démonstration}[part]
\newtheorem{definition}{Définition}[part]
\newtheorem{exemple}{Éxemple}[part]


\title{TIPE : Origami et constructibilité}

\begin{document}
\maketitle
\renewcommand{\contentsname}{Sommaire}
\tableofcontents{}

\part{Éléments d'introduction}
\chapter{Introduction à la constructibilité : règle et compas}
	
	\section{Définitions}
		\subsection{Outils d'algèbre}
			\begin{definition}[Corps]
			\end{definition}
			\begin{definition}[Espace Vectoriel]
			\end{definition}
			\begin{definition}[Extension de corp]
			\end{definition}
			\begin{definition}[Degré d'une extension]
			\end{definition}
			
		\subsection{Nombres algébriques}
			\begin{definition}[Nombre algébrique]
			\end{definition}
			\begin{definition}[Degré et polynôme minimal]
			\end{definition}
			\begin{proposition}
			\end{proposition}
			\begin{proposition}
			\end{proposition}
		
	\section{Ensemble des nombres constructibles}
		\subsection{Théorème de Wantzel}
			\begin{theorem}
				
			\end{theorem}
			\begin{proof}
				
			\end{proof}
		
		\subsection{Exemples célèbres}




\end{document}
