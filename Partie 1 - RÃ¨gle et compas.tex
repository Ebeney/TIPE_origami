\documentclass[a4paper,12pt,french]{report}  
\usepackage{babel} 
\usepackage[T1]{fontenc} 
\usepackage[utf8]{inputenc} 
\usepackage{lmodern}
\usepackage{amsthm}
\usepackage{stmaryrd}
\usepackage{amsmath}
\usepackage{amssymb}
\usepackage{mathrsfs}
\usepackage{geometry}
\usepackage{graphicx}
\usepackage{boiboites}
\geometry{hmargin=2.5cm,vmargin=2cm}

\newtheorem{ax}{Axiome}[section]
\newtheorem{lemma}{Lemme}[section]
%\newtheorem{proposition}{Proposition}[section]
\newtheorem{theorem}{Théorème}[section]
\newtheorem{definition}{Définition}[section]
\newtheorem{exemple}{Éxemple}[section]

% ------

\definecolor{violet}{rgb}{0.5,0,0.5}
\definecolor{orange}{rgb}{0.9,0.4,0}

\newboxedtheorem[boxcolor=blue,titleboxcolor = blue,titlebackground=blue!20,titlecolor = black]{proposition}{Proposition}{subsection}

% ------


\title{TIPE : Origami et constructibilité}

\begin{document}
\maketitle
\renewcommand{\contentsname}{Sommaire}
\tableofcontents{}

\chapter{Introduction à la constructibilité : règle et compas}
		Il est nécessaire de commencer notre étude en posant quelques définitions et résultats élémentaires qui nous seront utiles par la suite.
		\section{Outils d'algèbre}
			
			\begin{definition}[Corps] 
				Un corps est une structure algébrique \( (\mathbb{K} , + , \times ) \) telle que :
				\begin{enumerate}
					\item Les lois \( + \) et \( \times \) sont des applications \( \mathbb{K}^2 \longrightarrow \mathbb{K} \)
					\item \((\mathbb{K} , +)\) soit un groupe commutatif, c'est-à-dire :
						\[ 
							\left\{ 
							\begin{array}{lll}
								\exists \, 0_{\mathbb{K}} \in \mathbb{K} , \forall x \in \mathbb{K}, x + 0_{\mathbb{K}} = 0_{\mathbb{K}} + x = x
								\\
								\forall x \in \mathbb{K} , \exists k \in \mathbb{K} , x + k = k + x = 0_{\mathbb{K}} \mbox{  et on notera } -x = k
								\\
								\forall (x,y) \in \mathbb{K}^{2},
											x + y = y + x
										
								\\
								\forall (x, y, z) \in \mathbb{K}^3, (x + y) + z = x + (y + z)
									 
							\end{array}
							\right.
						\]
					\item \((\mathbb{K}\setminus\{0\}, \times) \) est un groupe commutatif :
						\[ 
								\left\{ 
								\begin{array}{lll}
									\exists \, 1_{\mathbb{K}} \in \mathbb{K}\setminus\{0\} , \forall x \in \mathbb{K}\setminus\{0\}, x \times 1_{\mathbb{K}} = 1_{\mathbb{K}} \times x = x
									\\
									\forall x \in \mathbb{K}\setminus\{0\} , \exists k \in \mathbb{K}\setminus\{0\} , x \times k = k \times x = 1_{\mathbb{K}} \mbox{  et on notera } x^{-1} = k
									\\
									\forall (x,y) \in (\mathbb{K}\setminus\{0\})^{2},
												x \times y = y \times x
									\\
									\forall (x, y, z) \in (\mathbb{K}\setminus\{0\})^3, (x \times y) \times z = x \times (y + z)
										 
								\end{array}
								\right.
							\]
					\item La loi \( + \) est distributive sur la loi \( \times \) :
						\[
						\forall (x, y, z) \in \mathbb{K}^3, x \times (y + z) = (y + z) \times x = xy + xz
						\]
				\end{enumerate}
			
			\end{definition}
			
			\begin{definition}[Corps engendré]
				Soit \(\mathbb{E}\) et \(\mathbb{F}\) deux corps, tel que \(\mathbb{E} \subset \mathbb{F}\), et soit \(x \in \mathbb{F} \). Le corps engendré par \(x\), noté \(\mathbb{E}(x)\), est le plus petit corps (au sens de l'inclusion) qui contient x et \(\mathbb{E}\).
			\end{definition}
			
			\begin{definition}[Sous anneau engendré par x]
				Soit \(\mathbb{E}\) et \(\mathbb{F}\) deux corps, tels que \(\mathbb{E}\subset\mathbb{F}\), et \(x\in\mathbb{F}\). Alors on appelle \emph{Sous anneau de F engendré par x} l'ensemble :
				\[\left\{y\in F / \exists P \in E[X], y = P(x)\right\}\]
				On le note \(E[x]\).{}
				
				\begin{proof}
					\(E[x]\) est l'image de l'anneau \(E[X]\) par le morphisme d'évaluation \[\varphi_{x} : \; P \mapsto P(x) \] donc c'est un anneau. Par la suite on introduira \(\varphi_{x}\) systématiquement, sans le redéfinir.
				\end{proof}
				
			\end{definition}
			
			
			\begin{definition}[Espace Vectoriel]
				Un \(\mathbb{K}\)-espace vectoriel \(E\)  est une structure algébrique \( (E, + , \cdot)\) telle que
				\begin{enumerate}
					\item \( (E, +) \) est un groupe commutatif
					\item La loi externe \( \cdot \) est une application \( \mathbb{K} \times E \longrightarrow E \) qui vérifie les axiomes suivants :
						\begin{enumerate}
							\item Pseudo-associativité :
								\(
								\forall (\lambda, \mu, x) \in \mathbb{K}^2 \times E, \mu(\lambda x) = (\mu \lambda) x
								\)
							\item Pseudo-distributivité :
								\(
								\left\{
									\begin{array}{ll}
										\forall (\lambda, \mu, x) \in \mathbb{K}^2 \times E , (\lambda + \mu)x = \lambda x + \mu x \\
										\forall (\lambda, x, y) \in \mathbb{K} \times E^2 , \lambda(x + y) = \lambda x + \lambda y
									\end{array}
								\right.
								\)
							\item Opérateur neutre :
								\(
								\forall x \in E, 1_\mathbb{K} \cdot x = x
								\)
						\end{enumerate}
				\end{enumerate}
				
			\end{definition}
			
			
			\begin{definition}[Extension de corps]
			
				Si \( \mathbb{K} \) et \( \mathbb{L} \) sont deux corps tels que \(\mathbb{K} \subset  \mathbb{L} \), alors
				\(\mathbb{L}\) est un \(\mathbb{K}\)-espace vectoriel. Dès lors, \(\mathbb{L}\) est appelée une \emph{extension} de \(\mathbb{K}\).
			
				\begin{proof}
					La structure d'espace vectoriel de \(\mathbb{L}\) s'hérite de sa structure de corps.
					\begin{enumerate}
						\item \(\mathbb{L}\) est un corps donc, en particulier, \((\mathbb{L}, +)\) est un groupe commutatif.
						\item Tout élément de \(\mathbb{K}\) étant en particulier un élément de \(\mathbb{L}\), il suffit de définir la loi
						externe \(\cdot\) à l'aide de la loi \( \times \) de \(\mathbb{L}\) pour que tous les axiomes de la loi externe d'un espace vectoriel s'héritent directement des propriétés de la loi \( \times \) d'un corps:
							\[
							\forall (x, y) \in \mathbb{K} \times \mathbb{L}, x \cdot y = x \times y
							\]
					\end{enumerate}
				\end{proof}
			\end{definition}
			
			
			
			
			\begin{definition}[Degré d'une extension]
				Soit \(\mathbb{L}\) une extension d'un corps \(\mathbb{K}\). Si la dimension de \(\mathbb{L}\) en tant que \(\mathbb{K}\)-espace vectoriel est finie, on l'appelle le \emph{degré} de l'extension, notée :
				\[
				[\mathbb{L}:\mathbb{K}] = \dim_\mathbb{K}\mathbb{L}
				\]
				S'il est égal à 2, on parlera d'\emph{extension quadratique}.
			\end{definition}
			
			\begin{proposition}[Relation de Chasles sur le degré]
				Soit \(K\), \(L\), \(M\) trois corps tels que \( K \subset L \subset M\), et les extensions sont de degré fini. On a alors :
				\[
				[M:K] = [M:L]{\times} [L:K]
				\]
			\end{proposition}
				\begin{proof}
					On pose tout d'abord :
					\[
					\left\{
						\begin{array}{ll}
							p = [M:L] \\
							n = [L:K]
						\end{array}
					\right.
					\]
					
					
					Ensuite, exhibons une base de \(M\) en tant que \(K\)-espace vectoriel afin d'obtenir sa dimension.
					Soit \((l_i)_{1 \leq i \leq n} \in L^n\) une base de \(L\) en tant que \(K\)-ev.
					Soit \((m_j)_{1 \leq j \leq p} \in L^n\) une base de \(M\) en tant que \(L\)-ev.
					
					Montrons que \((l_i m_j)_{(i, j) \in \llbracket 1, n \rrbracket \times \llbracket 1, p \rrbracket}\) est une base de M en tant que \(K\)-espace vectoriel.
					\begin{enumerate}
					\item \emph{Caractère générateur} : soit \(x \in M\), alors 
						\[
						\exists (\lambda_1, \dots, \lambda_p) \in L^p, \sum_{j  = 1}^{p} \lambda_j m_j = x
						\]
						Or \(\forall i \in \llbracket 1, p \rrbracket, \lambda_i \in L\), donc :
						\[ 
						\exists (\mu_{i, 1}, \dots, \mu_{i, n}) \in K^n, \sum_{j  = 1}^{n} \mu_{i, j} l_j = \lambda_i 
						\]
						Puis :
						\[
						x 	= \sum_{i  = 1}^{p} \lambda_i m_i  
							= \sum_{i  = 1}^{p} \left(\sum_{j  = 1}^{n} \mu_{i, j} l_j\right) m_i 
							= \sum_{(i, j) \in \llbracket 1, p \rrbracket \times \llbracket 1, n \rrbracket}\mu_{i, j} (l_j m_i)
						\]
					\item \emph{Caractère libre} : soit \((\lambda_{i, j})_{ (i, j) \in \llbracket 1, n \rrbracket \times \llbracket 1, p \rrbracket} \in K^{np}\) telle que :
							\[
							\sum_{(i, j) \in \llbracket 1, n \rrbracket \times \llbracket 1, p \rrbracket} \lambda_{i, j} (l_j m_i) = O
							\]
						Soit encore :
							\[
							\sum_{i=1}^{p} \underbrace{\left( \sum_{j = 1}^{n} \lambda_{i, j} l_j \right)}_{\in L}m_i = 0
							\]
						Donc, par liberté de \((m_1, \dots, m_p)\), on a :
							\[
							\forall i \in \llbracket 1, p \rrbracket, \sum_{j = 1}^{n} \underbrace{\lambda_{i, j}}_{\in K} l_j = 0
							\]
						Donc, par liberté de \((l_1, \dots, l_n)\), on obtient:
							\[
							\forall i \in \llbracket 1, p \rrbracket, \forall j \in \llbracket 1, n \rrbracket, \lambda_{i, j} = 0
							\]
					\item On a ainsi montré que \((l_i m_j)_{(i, j) \in \llbracket 1, n \rrbracket \times \llbracket 1, p \rrbracket}\) est une base du \(K\)-espace vectoriel \(M\). Or, cette base contient \(n\times p\) vecteurs, donc l'extension M est bien de degré \(n\times p\) par rapport à \(K\), ce qui achève la preuve. \( \)
					
					\end{enumerate}
				\end{proof}
    
    
    %----- Nombres algébriques -----%
		\section{Nombres algébriques}
			\begin{definition}[Nombre algébrique]
				Soit \(E\), \(F\), deux corps,  et \(x \in F\). On dit que \(x\) est \emph{algébrique} sur E si : 
				\[
					\exists P \in E[X]\setminus\{0\} , P(x) = 0
				\]
				Sinon, on dit que x est \emph{transcendant}
			\end{definition}
			
			\begin{exemple}
			$ $
			    \begin{enumerate} 
				    \item Le nombre \(\sqrt{2}\) est algébrique sur $\mathbb{Q}$, car il est racine du polynôme \(X^{2} - 2\)
				    \item Le nombre $i$ est algébrique sur $\mathbb{Q}$, car il est racine du polynome \(X^{2} +1\)
			    \end{enumerate}
			\end{exemple}
			\begin{definition}[Degré algébrique et polynôme minimal]
				Soit \(x \in F \) algébrique sur \(E\). Alors il existe un unique polynôme unitaire de degré minimal, non nul, dont x soit racine. On l'appelle le \emph{polynôme minimal}, et son degré est le \emph{degré algébrique} de \(x\).{}
			\end{definition}
				\begin{proof}
					Soit \(x \in F\), algébrique sur \(E\). On remarque que l'ensemble des polynômes qui s'annulent en \(x\) est exactement \(Ker \varphi_{x}\) qui est un sous groupe additif. De plus, 
					\[
						\begin{aligned}
							\forall (P,Q) \in Ker\varphi_{x} \times E[X], \; \varphi_{x}(PQ) &= \varphi_{x}(P)\varphi_{x}(Q) \\
											&=P(x)Q(x) \\
											&= 0
						\end{aligned}
					\]
					Donc \(Ker \varphi_{x}\) est absorbant pour la deuxième loi, donc c'est un idéal. Or \(E[X]\) est principal, donc il existe un unique polynôme unitaire \(P \in E[X]\) tel que \(Ker \varphi_{x} = P \cdot E[X]\)\\
					Comme \(x\) est algébrique, il existe un polynôme non nul qui s'annule en \(x\) donc \(Ker \varphi_{x}\) n'est pas réduit à \(\{0\}\), donc \(P \neq 0_{E[X]}\) \\
					Enfin, soit \(Q \in E[X]\setminus\{0\}\) tel que \( Q(x) = 0 \), alors \(Q \in Ker \varphi_{x} = P \cdot E[X] \) donc \(P | Q\), ce qui assure que le degré de P est minimal.\\
					Si de plus \(Q\) est supposé de degré minimal et unitaire, on a \(\exists \lambda \in E\setminus\{0\} , Q = \lambda P \) (la non-nulité vient du fait que \(Q\) est lui même non nul). Donc \(P\) et \(Q\) sont associés, et par unicité du polynome unitaire dans une classe d'association, on a \[P = Q\] 
				\end{proof}
			
			\begin{proposition}
				Soit E un corps, F un sur-corps de E. Soit L une E-algèbre contenant E, inclut dans F, de dimension finie. Si \(x \in L\), alors x est algébrique sur E. En particulier, le résultat est vrai si L est une extension de corps
			\end{proposition}
				\begin{proof}
					Posons \(n = dim_{E}(L) \in \mathbb{N} \). Soit \(x \in L \). Considérons la famille
					\[{}
						\left( 1 , x , x^{2} , \dots , x^{n} \right)
						\]
					\begin{enumerate}
						\item Soit certains éléments apparaissent plusieurs fois, donc la famille est liée
						\item Soit tous les éléments sont différents, et comme famille a \(n+1\) éléments dans un espace de dimension \(n\), la famille est liée.
					\end{enumerate}
					Dans tous les cas, la famille est liée et on a : 
						\[{}
							\exists (\lambda_{0},\dots,\lambda_{n}) \in (E^{n})\backslash {(0,\dots,0)}, \sum_{i = 0}^{n} \lambda_{i}x^{i} = 0
						\]
						On pose alors : \(P = \sum_{i = 0}^{n} \lambda_{i}X^{i} \) qui n'est pas le polynome nul, car \( (\lambda_{0},\dots,\lambda_{n}) \) n'est pas la famille nulle. Et on a \(P(x) = 0 \), donc x est algébrique  \( \)
				\end{proof}
			
			\begin{proposition}
				Soit \(x \in F \), algébrique sur \(E\), alors \(E(x)\) est de dimension finie sur \(E\), et le degré de cette extension est égal au degré algébrique de \(x\){}
			\end{proposition}
				\begin{proof}
					Soit \(x \in F \), algébrique sur \(E\).{}
					
	\begin{enumerate}
		\item Montrons tout d'abord que $E[x]$ est une E-algèbre de dimension finie.\\ 
						Soit \(P_{0} \in E[X]\setminus\{0\} \) le polynôme minimal de $x$, notons \(n = deg(P_{0})\).
					Posons : 
					\[{}
						\begin{array}[t]{lccl}
							\varphi : 
							& E_{n-1}[X] & \longrightarrow & E[x] \\
							& P & \longmapsto & P(x)
						\end{array}
					\] et montrons que c'est un morphisme bijectif.
					
					\begin{enumerate}
						\item $\varphi$ est un morphisme comme restriction du morphisme d'évalutation $\varphi_{x}$
						\item \emph{Injectivité} - Soit \(P \in Ker(\varphi)\)et supposons que \(P \neq 0_{E_{{n_-1}}[X]} \). Quite à considérer le coefficient dominant de P, noté \(p\), et le polynôme \(\frac{1}{p}P\) unitaire, on peut supposer sans perte de généralité que P est unitaire. Comme \(P\) appartient à \(E_{n-1}[X]\), on a \(deg(P) < n\). Or \(P(x) = 0\), donc P contredit le choix de \(P_{0}\), ce qui est absurde. Donc P est le polynome nul, et \(\varphi\) est bien injectif
						
						\item \emph{Surjectivité}
						Soit $y \in E[x]$, on a : $\exists A \in E[X], A(x) = y$. Par division euclidienne : 
						\[{}
							\exists (Q,R) \in E[X] \times E_{n-1}[X] , A = PQ + R
						\]
						Donc, on a :
						\[{}
						\begin{aligned}
							y &= A(x) \\
							&= (PQ + R)(x)\\
							&=\underbrace{P(x)}_{=0}Q(x) + R(x)\\
							&=\varphi(R)
						\end{aligned}
						\]
						Donc $\varphi$ est bien surjective
					\end{enumerate}
				Finalement, par propriété des isomorphismes, on déduit que $E[x]$, comme espace isomorphe à un espace de dimension finie, est lui même de dimension finie, et :
				\[{}
				dim(E[x]) = dim(E_{n-1}[X]) = n = deg(P_{0})
				\]
		
		\item Montrons maintenant que $E[x]$ est un corps. Pour cela, il suffit de montrer que tout élément non nul de  $E[x]$ est invesible, les autres propriétés sont directement héritées de la structure d'algèbre.
			Soit $y \in E[x]\setminus\{0\} $. Comme élément d'une E-algèbre de dimension finie, par la proposition précédente, $y$ est algébrique sur $E$. Soit \(P_{1} \in E[X]\) son polynôme minmal, notons \(P = \sum_{i=0}^{m}a_{i}X^{i}\).{}
			\begin{enumerate}
				\item Supposons $a_{0} = 0$ , alors on aurait : $P_{1} = X\times\sum_{i=1}^{n}a_{i}X^{i-1}$ donc comme $y \neq O$, par intégrité,
				\[{}
					\sum_{i=1}^{n}a_{i}y^{i-1} = 0
				\]
				Donc $P_{1}$ n'est pas de degré minimal, absurde. Donc : $a_{0} \neq 0 $
			
				\item On a en factorisant par y :
				\[{}
					\begin{aligned}
						P_{1}(y) = 0 &\Rightarrow \sum_{i=1}^{n}a_{i}y^{i} = -a_{0}\\
									&\Rightarrow \sum_{i=1}^{n}a_{i}y^{i-1} = \frac{-a_{0}}{y}\\
									&\Rightarrow \underbrace{\frac{-1}{a_{0}}\sum_{i=1}^{n}a_{i}y^{i-1}}_{\in E[x]} = \frac{1}{y}
					\end{aligned}
				\]
			\end{enumerate}
			Ainsi $y$ est inversible, et $E[x]$ est un corps.
			
		\item Montrons finalement que $E[x] = E(x)$. 
			\begin{enumerate}
				\item On a comme $E(x)$ est un corps, il est stable par produit, combinaisons linaires, donc \(\forall P \in E[X], P(x) \in E(x)\). Or \[ \forall y \in E[x], \exists P_{y} \in E[X], P(x) = y \] Donc,  \[E[x] \subset E(x)\]
				\item 
				Par construction de $E(x)$, tout corps contenant $E$ et $x$ contient \(E(x)\), donc, 
				\[ E[x] \supset E(x) \]
			\end{enumerate}
	\end{enumerate}
	Finalement \(E(x) = E[x]\), et c'est bien une extension de corps, de dimension finie, du degré algébrique de x.
				\end{proof}
				
		
		\begin{proposition}
			Soit $E,F$ deux corps, et $x,y \in F$, algébriques sur E. Alors, 
			$x+y$ et $xy$ sont aussi algébriques sur E. Enfin si $x\neq0$, $\frac{1}{x}$ est algébrique sur $E$. 
		\end{proposition}
		
		\begin{proof}
			Soit \(x,y \in F\), algébriques sur $E$. On note \(P_{x},P_{y} \in E[X]\) leur polynômes minimaux.
			\begin{enumerate}
				\item
				    Comme on a vu, $E(x)$ est un corps, donc il est légitime de considérer $(E(x))(y)$.
				    De plus $y$ est algébrique sur $(E(x))(y)$, car \(P_{y} \in E[X] \subset (E(x))[X]\), donc $(E(x))(y)$ est de dimension finie sur $E(x)$.
				\item 
				    Soit \[\mathcal{B} = (e_{1}, \cdots, e_{n}) \] une base de $E(x)$ (comme $E$-espace vectoriel), et \[\mathcal{C} = (\epsilon_{1},\cdots, \epsilon_{m})\] une base de $(E(x))(y)$ (comme $E(x)$-espace vectoriel).\\
				    Soit \(a \in (E(x))(y) \), on a : \[\exists (\lambda_{1},\cdots,\lambda_{m}) \in (E(x))^{m}, \; a = \sum_{i=1}^{m}\lambda_{i}\epsilon_{i}\]
				    Or comme $\mathcal{B}$ est en particulier génératrice dans $E(x)$, on a:
				    \[\forall i \in \llbracket 1 , m \rrbracket, \exists (\alpha_{i,1},\cdots,\alpha_{i,n}) \in E^{n}, \lambda_{i} = \sum_{j=1}^{n}\alpha_{i,j}e_{j} \]
				    Donc, finalement, 
				    \[a = \sum_{i=1}^{m} \left(  \sum_{j=1}^{n}\alpha_{i,j}e_{j} \right)\epsilon_{i} \]
				    Ainsi la famille \((e_{j}\epsilon_{i})_{(j,i) \in \llbracket 1 , n \rrbracket \times \llbracket 1 , m \rrbracket} \) est génératrice dans $(E(x))(y)$ vu comme {$E$-espace vectoriel}.
				\item{}
				    Or cette famille est finie, donc, $(E(x))(y)$ est de dimension finie sur $E$.  Or, \\\({x+y,xy \in (E(x))(y)}\), par la proposition précédemment démontrée, $x+y$ et $xy$ sont algébriques sur E. Enfin si $x \neq 0, \frac{1}{x} \in (E(x))(y)$ est aussi algébrique sur $E$.
			\end{enumerate}
		\end{proof}
		
		
		\begin{proposition}
			Les nombres algébriques sur E forment un corps.
		\end{proposition}
		\begin{proof}
		    C'est immédiat avec les résultats précédents.
		\end{proof}
	
		\section{Application à $\mathbb{Q}$}
	En ayant mis en place tous les outils nécessaires, on peut désormais aboutir à quelques résultats plus spécifiques à $\mathbb{Q}$, qui seront utiles pour établir avec précision la condition nécessaire et suffisante pour qu'un nombre soit constructible.
		
		\begin{proposition}
			Soit E une extension quadratique sur $\mathbb{Q}$. Alors
			\[{}
				\exists\, \delta \in \mathbb{Q}, E = \mathbb{Q}(\sqrt{\delta})
			\]
		\end{proposition}
		 Par la suite, on notera indifferemment $\mathbb{Q}(\sqrt{\delta})$ ou $\sqrt{\delta}\,\mathbb{Q}$
		\begin{proof}
			Soit E une extension quadratique sur $\mathbb{Q}$. Complétons la famille libre $(1_{\mathbb{Q}})$ en une base : $(1_{\mathbb{Q}},e)$. Comme élément d'une extension finie, $e$ est algébrique, on note $d$ son degré algébrique.
			\begin{enumerate}
				\item  Montrons que $d = 2$
				\begin{enumerate}
				    \item On a que $d\geq2$, car sinon $e$ serait racine d'un polynome de la forme $aX +b$, avec $(a,b) \in \mathbb{Q}^{2}$, donc serait rationnel. Or  $e \notin Vect(1_{\mathbb{Q}}) = \mathbb{Q}$ par liberté de la base. Donc \[d\geq 2\]
				    \item On a que $\mathbb{Q}(e) \subset E $ car une extension quadratique est en particulier un corps. En observant les dimensions, on a $d \leq dim_{\mathbb{Q}}(E) = 2$. Donc 
				    \[d \leq 2\] On a ainsi montré que $d=2$, et l'inclusion $\mathbb{Q}(e) \subset E $ avec l'égalité des dimensions donne : $E = \mathbb{Q}(e)$.
				\end{enumerate}
				
				\item  Soit $P = aX^{2} + bX +c \in \mathbb{Q}[X]$ le polynôme minimal de $e$. On a $a\neq 0$ sans quoi $e$ ne serait pas de degré algébrique 2. Donc \[e = \frac{-b \pm \sqrt{b^{2} - 4ac}}{2a}\]
				Posons $\delta = b^{2} - 4ac$ et montrons que $E = \mathbb{Q}(\sqrt{\delta}$.\\
				On a immédiatement que la famille \((1_{\mathbb{Q}},\sqrt{\delta})\) est libre, sans quoi $\sqrt{\delta}$ serait rationnel et $e$ aussi. Donc c'est une base, car E est de dimension 2.\\
				En réutilisant le résonnement du 1), avec $e' = \sqrt{\delta}$, on obtient : \[E = \mathbb{Q}(\sqrt{\delta})\]
				    
				    
			\end{enumerate}
		\end{proof}

\end{document}
